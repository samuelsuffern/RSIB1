\documentclass{article}
\usepackage[utf8]{inputenc}

\usepackage[spanish,es-nodecimaldot]{babel}
\usepackage[left=2cm,right=2cm,top=2cm,bottom=2cm]{geometry}
\usepackage{amsmath}

\usepackage{natbib}
\usepackage{subfigure}
\usepackage{graphicx}
\usepackage{mcode}
\definecolor{mygreen}{RGB}{0,100,0}
\usepackage{color}   %May be necessary if you want to color links
\usepackage{hyperref}
\hypersetup{
    colorlinks=true, %set true if you want colored links
    linktoc=all,     %set to all if you want both sections and subsections linked
    linkcolor=blue,  %choose some color if you want links to stand out
}

\lstset{language=Matlab,%
    %basicstyle=\color{red},
    breaklines=true,%
    morekeywords={matlab2tikz},
    keywordstyle=\color{blue},%
    morekeywords=[2]{1}, keywordstyle=[2]{\color{black}},
    identifierstyle=\color{black},%
    stringstyle=\color{purple},
    commentstyle=\color{mygreen},%
    showstringspaces=false,%without this there will be a symbol in the places where there is a space
    numbers=left,%
    numberstyle={\tiny \color{black}},% size of the numbers
    numbersep=9pt, % this defines how far the numbers are from the text
    emph=[1]{for,switch,case,end,break},emphstyle=[1]\color{blue}, %some words to emphasise
    %emph=[2]{word1,word2}, emphstyle=[2]{style},  
    }
\begin{document}
\begin{titlepage}
\centering
{\scshape\LARGE Práctica B1 RSI\par}
	\vspace{1cm}
	{\scshape\Large Parte 1. \par}
	\vspace{1.5cm}
	{\huge\bfseries Procesado de variaciones rápidas de señal debidas a multitrayecto\par}
	\vspace{2cm}
	{\Large\itshape Samuel John Suffern Sánchez\par}
	\vfill
	\vfill
% Bottom of the page
	{\large \today\par}
\end{titlepage}


%Indice
\tableofcontents
\listoffigures
\pagebreak




\section{Inicio del entorno en Matlab}
    \par Los gráficos y resultados de esta práctica se obtienen con la función \mcode{pr10.mat}. 
    \par Lo primero que haremos será cargar en nuestro workspace el fichero: \mcode{series11.mat}
    \begin{lstlisting}
        load series11.mat
    \end{lstlisting}

    \par Contiene una señal que a tiene en la primera columna el eje de tiempo y en la segunda, la potencia recibida en dBm. El eje de tiempos está muestreado cada 
    \( t_{s} = 0.05 \hspace{0.1cm} s\), es decir,
    \( f_s = \frac{1}{t_s} = 20 \hspace{0.1cm} Hz\). Existe un muestreo uniforme. La resta entre dos muestras consecutivas, me da el muestreo que estamos utilizando.
    \begin{lstlisting}
        tAxix = series11(:,1); % eje temporal
        ts = tAxix(2)-tAxix(1); % muestreo del eje de tiempos
        PdBm = series11(:,2);
        figure, plot(tAxix, PdBm)
        xlabel('Elapsed time, s')
        ylabel('Received power, dBm');title('Potencia recibida [dBm]');
    \end{lstlisting}
    \par Las líneas de código muestran la forma de la señal recibida a lo largo del tiempo en unidades logarítmicas .
    \begin{figure}[h]
        \centering
        \includegraphics[scale=0.55]{figures_outputs/B1/series11.eps}
        \caption{Potencia recibida en unidades logarítmicas}
        \label{fig:series_11}
    \end{figure}

    \par La señal ha sido recibida en un instrumento de medida o receptor de prueba, cuando el transmisor emite una portadora sin modular a una frecuencia \(f_0=2.0  \hspace{0.1cm}GHz \). El receptor se supone que tiene una impedancia \( Z=R=50\Omega\).
    \par Los valores medidos están en dBm (dB relativos a 1 mW).
    \par En Matlab, podemos obtener la misma gráfica pero en unidades naturales.
    \begin{lstlisting}
        pW = 10.^(PdBm/10)/1000;   % divide by 1000 to go from mW to W
        figure, plot(tAxix, pW)
        xlabel('Elapsed time, s')
        ylabel('Received power, W');title('Potencia recibida [W]');
    \end{lstlisting}
    \par En la siguiente carilla observamos la gráfica de la señal a lo largo del tiempo en unidades lineales.
    \clearpage
    \begin{figure}[h]
        \centering
        \includegraphics[scale=0.55]{figures_outputs/B1/p_output_lineal.eps}
        \caption{Representación de la potencia recibida en unidades naturales}
        \label{fig:received_power_unidades_naturales}
    \end{figure}
\section{Estudio de la tensión}
    \par Utilizando las expresiones del capítulo 2 de Teoría, las variaciones de voltaje se podían expresar como:
    \begin{equation}
        \tag{amplitud de la señal RF}
        v = \sqrt{2p_r}|h_0|
        \label{eq:voltaje_entrada}
    \end{equation}
    con \(p_r : Potencia \hspace{0.1cm}media\hspace{0.1cm} recibida\) y \( |h_0| : variaciones\hspace{0.1cm} aleatorias \hspace{0.1cm}del \hspace{0.1cm}canal\).
    \par V, es función de \(|h_0|\), variable aleatoria que representa la envolvente de la señal y sabemos que se puede caracterizar mediante una distribución Rayleigh. Por tanto V, seguirá también una distribución Rayleigh con función densidad de probabilidad: 
    \begin{equation}
    \tag{función densidad de probabilidad}
        f_v(v) = \frac{v}{p_r}exp\left({-\frac{v^2}{2p_r}}\right)
        \label{eq:fdp}
    \end{equation}
    \par En Clase de Teoría (Grupos A) supusimos que la impedancia era \(R=1 \) Modificamos la expresión anterior para tener en cuenta la impedancia:
    \begin{equation}
    \tag{voltaje a la entrada del receptor}
        v = \sqrt{2p_rR}|h_0|
        \label{eq:voltaje_entrada_R}
    \end{equation}
    \subsection{Tensión en Matlab}
        \par En matlab por tanto será de la siguiente manera:
        \begin{lstlisting}
            R = 50; %Impedancia
            v = sqrt(2*pW*R); % Convierto a voltios
            figure, plot(tAxix,v)
            xlabel('Elapsed time, s')
            ylabel('Voltage at Rx input, V');title('Tensión a la entrada del receptor [V]')
        \end{lstlisting}
        \par En la siguiente carilla observamos la tensión de la señal recibida en unidades lineales sin normalizar.
        \clearpage
        \begin{figure}[h]
            \centering
            \includegraphics[scale=0.55]{figures_outputs/B1/v_output.eps}
            \caption{Tensión en unidades lineales}
            \label{fig:v_output}
        \end{figure}
        \subsubsection{Función densidad de probabilidad}
            \par Su función de densidad de probabilidad en función de la impedancia:
            \begin{equation}
            \tag{función densidad de probabilidad (R)}
                f_v(v) = \frac{v}{Rp_r}exp\left({-\frac{v^2}{2Rp_r}}\right)
                \label{eq:fdp_R}
            \end{equation}
    \subsection{Tensión normalizada en Matlab}
        \par Definiendo ahora una nueva variable aleatoria,normalizada con respecto a la potencia media:
        \begin{equation}
        \tag{voltaje a la entrada del receptor normalizado}
            v' =\frac{v}{ \sqrt{2p_rR}} = |h_0|
            \label{eq:voltaje_entrada_normalizado}
        \end{equation}
        \par En matlab serían las siguientes líneas de código:
        \begin{lstlisting}
            vnorm = v / sqrt(2*meanPw*R); %normalizo
            figure, plot(tAxix, vnorm)
            xlabel('Elapsed time, s')
            ylabel('Normalized voltage (linear)'); title('Voltaje normalizado');
        \end{lstlisting}
        \begin{figure}[h]
            \centering
            \includegraphics[scale=0.55]{figures_outputs/B1/v_output_norm.eps}
            \caption{Voltaje normalizado}
            \label{fig:my_label}
        \end{figure}
        \clearpage
        \subsubsection{Función densidad de probabilidad y Función de distribución}
            \par En cuanto a su \textbf{función densidad de probabilidad}:
            \begin{equation}
                \tag{función densidad de probabilidad normalizada}
                f_v(v') = 2v'e^{-v'^2}
                \label{eq:fdp_norm}
            \end{equation}
            \textbf{Función de distribución}:
            \begin{equation}
                \tag{Función de distribución normalizada}
                P(v'\leq U) = F_{v'}(U)=1-exp(-U^2)
                \label{eq:FD_norm}
            \end{equation}
            \par Probabilidad que mi voltaje normalziado esté por debajo de un determinado valor.
            \par Recordemos como son las formas teóricas de \ref{eq:fdp_norm} y \ref{eq:FD_norm}. Para ello hacemos los siguientes cálculos en Matlab.
            \par Queremos verificar que v' ( en Matlab \mcode{vnorm} ) sigue una distribución Rayleigh.:
            \begin{lstlisting}
                %Voy a pintar la fdp normalizada
                Max = 3.0;
                vnormAxis = [0:0.01:Max]; % Creo un eje para pintar, no debo usar Vnorm, son valroes desordenado
                fdp = 2*vnormAxis.*exp(-vnormAxis.^2);
                FD = 1-exp(-vnormAxis.^2);
                figure, plot(vnormAxis, fdp, vnormAxis, FD, '--', 'LineWidth',2)
                title('Theoretical, rms = 1 Rayleigh pdf and CDF')
                xlabel('Normalized voltage (linear)')
                ylabel('Probability')
                legend('pdf','CDF')
                grid
            \end{lstlisting}  
            \par Observo que la salida se corresponden con los valores teóricos de la fpd y FD del voltaje normalizado.
            \begin{figure}[h]
                \centering
                \includegraphics[scale=0.55]{figures_outputs/B1/rayleigh_fpd.eps}
                \caption{Función densidad y distribución de probabilidad de la tensión normalizada calculada con Matlab}
                \label{fig:fdp_norm}
            \end{figure}
    \subsection{Relación forma teórica respecto a la experimental}
        \subsubsection{Función densidad de probabilidad teórica vs experimental}
            \par Vamos a relacionar la función densidad de probabilidad teórica con la experimental.
            \begin{lstlisting}
                Nbins = 20; %Nbins por defecto es 20
                [a,b] = hist(vnorm, Nbins);
                a = a/length(vnorm);%Probabilidad estimada
                histBin = b(2)-b(1);
                figure, bar(b,a,'y'), hold on
                plot(vnormAxis, fdp*histBin,'r','LineWidth',2)
                plot(vnormAxis, fdp,'.r','LineWidth',2)
                title('Theoretical, rms = 1 Rayleigh pdf and experimental pdf')
                xlabel('Normalized voltage (linear)')
                ylabel('Probability')
                legend('Histogram','pdf x histBin','pdf')
                xlim([0 Max])
            \end{lstlisting}
            \begin{figure}[h]
                \centering
                \includegraphics[scale=0.60]{figures_outputs/B1/fpd_teoric_vs_exp.eps}
                \caption{Relación entre la forma teórica y la experimental de la fdp }
                \label{fig:fdp_teor_vs_exp}
            \end{figure}
        \subsubsection{Función de Distribución teórica vs experimental}
            \par Ahora queremos observar la función de distribución teórica con respecto a la experimental.
            \begin{lstlisting}
                aa = cumsum(a);
                figure, bar(b, aa,'y'), hold
                plot(vnormAxis, FD, 'r', 'LineWidth',2)
                title('Theoretical, rms = 1 Rayleigh CDF and experimental CDF')
                xlabel('Normalized voltage (linear)')
                ylabel('Probability')
                xlim([0 Max])
            \end{lstlisting}
            \par La función de distribución experimental se aproxima mucho a la calculada teóricamente.
            \begin{figure}[h]
                \centering
                \includegraphics[scale=0.55]{figures_outputs/B1/FD_teor_vs_exp.eps}
                \caption{Función de distribución teórica con respecto a la experimental}
                \label{fig:FD_teór_vs_exp}
            \end{figure}
            \par Una vez estudiado el efecto de la tensión, pasamos a ver el comportamiento en potencia.
\section{Estudio de la potencia}
    \par Vamos a realizar el estudio de la potencia y vamos a ver si coinciden las funciones densidad y distribución de probabilidad teóricas con la experimental.
    \par Para empezar vamos a definir la potencia:
    \begin{equation}
        \tag{Variable aleatoria de la potencia}
        P = \frac{v^2}{2}
    \end{equation}
    \par No añadimos la impendancia debido a que ya está incluido en el término \(v\).
    \subsection{Representación de la potencia en Matlab}
        \begin{lstlisting}
            p = (v.^2)/2; % Defino la potencia en lugar del voltaje
            figure, plot(tAxix, p)
            xlabel('Elapsed time, s')
            ylabel('Power (linear)')
        \end{lstlisting}
        \par Observamos la forma de la potencia en función del tiempo.
        \begin{figure}[h]
            \centering
            \includegraphics[scale=0.55]{figures_outputs/B1/p_output_lineal.eps}
            \caption{Potencia en unidades lineales}
            \label{fig:pot_unidades_lineal}
        \end{figure}
        \subsubsection{Función densidad de probabilidad}
            \par En este caso se tiene que la potencia sigue una distribución exponecial con función densidad de probabilidad:
            \begin{equation}
                \tag{Función densidad de probabilidad de la potencia}
                f_p (P) = \frac{1}{P_r}e^{-\frac{P}{P_r}}
            \end{equation}
    \subsection{Representación de la potencia normalizada en Matlab}
        \par Ahora vamos a normalizar para obtener una nueva variable aleatoria:
        \begin{equation}
            P' = \frac{P}{P_r \cdot R}
        \end{equation}
        \par Si ahora observamos la potencia normalizada con respecto a al tiempo:
        \begin{lstlisting}
            pnorm = p / (mean(pW)*R); %normalizo
            figure, plot(tAxix, pnorm)
            xlabel('Elapsed time, s')
            ylabel('Normalized power (linear)')
        \end{lstlisting}
        \begin{figure}[h]
            \centering
            \includegraphics[scale=0.5]{figures_outputs/B1/p_output_normal.eps}
            \caption{Potencia normalizada en unidades lineales}
            \label{fig:pot_unidades_lineal}
        \end{figure}
        \subsubsection{Función de densidad y distribución de la potencia normalizada}
            \par Vamos a estudiar de manera conjunta la función densidad de probabilidad y la función de distribución teóricas de la potencia normalizada. Para ello utilizamos las siguientes expresiones:
            \par \textbf{Función densidad de probabilidad}:
                \begin{equation}
                    \tag{Función densidad de probabilidad}
                    f_p (P') = e^{-P'}
                \end{equation}
            \par \textbf{Función de distribución}:
                \begin{equation}
                    \tag{Función de distribución}
                    F_{p'}(U) = P(P' \leq U ) = 1-e^{-U}
                \end{equation}
            \par Ahora aplicamos las ecuaciones utilizando matlab:
            \begin{lstlisting}
                Max = 10.0;
                pnormAxis = [0:0.01:Max]; % Creo un eje para pintar, no debo usar Vnorm, son valroes desordenado
                fdp_p = exp(-pnormAxis);
                FD_p = 1-exp(-pnormAxis);
                figure, plot(pnormAxis, fdp_p, pnormAxis, FD_p, '--', 'LineWidth',2)
                title('Theoretical, mean = 1 Exponential pdf and CDF')
                xlabel('Normalized voltage (linear)');ylabel('Probability');legend('pdf','CDF')
            \end{lstlisting}
            \par La salida corresponde a las curvas función de distribución y función densidad de probabilidad teóricas de la potencia.
            \begin{figure}[h]
                \centering
                \includegraphics[scale=0.5]{figures_outputs/B1/exponencial_teorica.eps}
                \caption{Curvas exponencial de la función densidad de probabilidad y función de distribución teóricas de la potencia}
                \label{fig:my_label}
            \end{figure}
        \subsubsection{Función de densidad de probabilidad teórica vs experimental}
            \par Una vez conocemos como es la curva teórica de la fdp, tenemos que comprobar si la curva experimental se acerca o no.
            \begin{lstlisting}
                Nbins = 20; 
                [a,b] = hist(pnorm, Nbins);
                a = a/length(pnorm);
                histBin = b(2)-b(1);
                figure, bar(b,a,'y'), hold on
                plot(pnormAxis, fdp_p*histBin,'r','LineWidth',2)
                plot(pnormAxis, fdp_p,'.r','LineWidth',2)
                title('Theoretical, mean=1 exponential pdf and experimental pdf')
                xlabel('Normalized voltage (linear)')
                ylabel('Probability')
                legend('Histogram','pdf x histBin','pdf')
                xlim([0 Max])
             \end{lstlisting}
             \par Obtenemos la siguiente salida:
        \begin{figure}[h]
            \centering
            \includegraphics[scale=0.6]{figures_outputs/B1/fdp_teor_vs_exp.eps}
            \caption{Similitud entre la función densidad de probabilidad teórica y experimental}
            \label{fig:fpd_teor_vs_exp_exponential}
        \end{figure}
        \par Observamos que las probabilidades obtenidas son bastante bajas y no se aproximan al valor teórico. Si quisiemos mejorar esto, basta con hacer la derivada:
        \begin{lstlisting}
             Nbins = 20; 
                [a,b] = hist(pnorm, Nbins);
                a = a/length(pnorm);
                a=[0.0 , a];
                histBin = b(2)-b(1);
                aAcum = cumsum(a);
                aAprox = diff(aAcum)/histBin;
                figure, bar(b,aAprox,'y'), hold on
                plot(pnormAxis, fdp_p,'.r','LineWidth',2)
                title('Theoretical, mean=1 exponential pdf and experimental pdf')
                xlabel('Normalized voltage (linear)')
                ylabel('Probability')
                legend('Histogram','pdf')
                xlim([0 Max])
        \end{lstlisting}
        \par Obtendríamos la siguiente salida:
        \clearpage
         \begin{figure}[h]
         \centering
         \includegraphics[scale=0.55]{figures_outputs/B1/fdp_p_close.eps}
         \caption{Aumento de la probabilidad al hacer la media de los valores tomados en el histograma}
         \label{fig:my_label}
     \end{figure}
     \par Podemos observar una mejoría en la aproximación haciendo una derivada discreta. Esta verifiación a ojo no vale, tendríamos que hacer la prueba de chi cuadrado.
    \par Si ahora comparamos función de distribución teórica con respecto a la experimental:
    \begin{lstlisting}
        [a,b] = hist(pnorm, Nbins);
        a = a/length(pnorm);
        aa = cumsum(a);
        figure, bar(b, aa, 'y'),hold
        plot(pnormAxis, FD_p, 'r', 'LineWidth',2)
        title('Theoretical, mean = 1 Exponential CDF and experimental CDF')
        xlabel('Normalized voltage (linear)')
        ylabel('Probability')
        xlim([0 Max])
    \end{lstlisting} 
     
    \begin{figure}[h]
        \centering
        \includegraphics[scale=0.55]{figures_outputs/B1/FD_comparacion.eps}
        \caption{Comparación función distribución teórica vs experimental}
        \label{fig:my_label}
    \end{figure}
    \par Aproximación bastante ajustada.
    \clearpage
\section{Superposición de las funciones}
    \par Por último vamos a representar en la misma gráfica superpuestas, las pdfs y CDFs de las dos series v' y p' con el eje-x en dB.Basta con escribir las siguientes líneas de código:
    \begin{lstlisting}
        figure,
        vnormdb= 20*log10(vnormAxis);
        pnormdb=10*log10(pnormAxis);
        semilogy(vnormdb,fdp);hold on;
        semilogy(pnormdb,fdp_p);hold off;title('Superposición pdfs');ylabel("Probability");xlabel("dB");legend('Pdf tensión normalizada','Pdf potencia normalizada')
    \end{lstlisting}
    \begin{figure}[h]
        \centering
        \includegraphics[scale=0.55]{figures_outputs/B1/fpd_pot_tens.eps}
        \caption{Superposición funciones densidad de probabilidad teóricas de la tensión y potencia normalizadas}
        \label{fig:my_label}
    \end{figure}
    \par Ahora vemos la superposición de las funciones distribución:
    
    \begin{lstlisting}
        semilogy(vnormdb,FD);hold on;
        semilogy(pnormdb,FD_p);hold off;title('Superposición FDs');ylabel("Probability");xlabel("dB");
        legend('FD tensión normalizada','FD potencia normalizada'); 
    \end{lstlisting}
    \begin{figure}[h]
        \centering
        \includegraphics[scale=0.55]{figures_outputs/B1/FD_pot_tension.eps}
        \caption{Superposición funciones distribución de probabilidad teóricas de la tensión y potencia normalizadas}
        \label{fig:my_label}
    \end{figure}      
    \par Se superponen una encima de la otra debido a la normalización que se le hacen a las variables aleatorias  \( v' \) y \( p' \)
    \par 

\end{document}

